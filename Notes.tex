\documentclass[11pt,a4paper]{article}

\usepackage[utf8]{inputenc}
\usepackage{parskip}
\usepackage{tabularx}
\usepackage{amsmath}
\usepackage{amssymb}
\usepackage{geometry}
\usepackage{booktabs}
\usepackage{hyperref}
\geometry{a4paper, left=20mm, right=20mm, top=20mm, bottom=20mm}

\usepackage{fancyhdr}
\pagestyle{fancy}
\lhead{Anthony Catterwell}
\chead{\textsc{University of Edinburgh}}
\rhead{Linear Programming}

\title{Linear Programming, Modelling \& Solution}
\author{Anthony Catterwell}


\begin{document}
\maketitle
% \tableofcontents

% \break\

\section*{Derivation}
We have
\begin{align*}{}
    f = \overline{\textbf{c}}^{T}\textbf{x} &=
    \textbf{c}_B^T \textbf{x}_B + \textbf{c}_N^T \textbf{x}_N \\
    \overline{A}\textbf{x} = \textbf{b} &= B\textbf{x}_B + N\textbf{x}_N
\end{align*}
combining the two gives us:
\begin{align*}{}
    f &= \widehat{f} + \widehat{\textbf{c}}_N^T \textbf{x}_N \\
      &= \{ \textbf{c}^T_B \widehat{\textbf{b}} \} +
      \{ \textbf{c}_N - N^T B^{-T} \textbf{c}_B \} \; \textbf{x}_N \\
      &= \{ \textbf{c}_B^T \, B^{-1}\textbf{b} \} +
      \{ \textbf{c}_N - N^T B^{-T} \textbf{c}_B \} \; \textbf{x}_N
\end{align*}
where
\[
    \widehat{\textbf{c}}_N = \textbf{c}_N - {N^T} {B^{-T}} \textbf{c}_B
\]
\[
    \widehat{f} = \textbf{c}_B^T\widehat{\textbf{b}}
\]
$\widehat{f}$ is the objective value when $\textbf{x}_N = \textbf{0}$
(so $\textbf{x}_B = \widehat{\textbf{b}}$)\\

\section*{LP Theory: Analysis of LPs in standard form}

\subsection*{Definition: Feasible Vertex I}
The vertex of the feasible region $K$ is a point $\textbf{x} \in K$ which does not lie strictly
within any line segment joining two points in $K$.

\subsection*{Theorem 1: A unique optimal solution is a vertex}
If an LP has a unique optimal solution then it is a vertex.

\subsection*{Theorem 2: Non-unique optimal solution at a vertex}
If an LP has a non-unique optimal solution then there is an optimal solution at a vertex.

\subsection*{Definition: Feasible Vertex II}
A vertex of the feasible region $K$ is a point $\textbf{x} \in K$ with

\begin{itemize}
    \item $n$ zero components
    \item $m$ non-negative components uniquely defined by $\overline{A}\textbf{x} = \textbf{b}$
\end{itemize}

\section*{LP Theory: Basic feasible solutions and optimality conditions for LP problems}

\subsection*{Definition: A basic solution}
The point $\textbf{x} \in \mathbb{R}^{n+m}$ is a \textbf{basic solution} of an LP problem in
standard form if there is a \textbf{partition} of $\{1, 2, \ldots, n+m\}$ into

\begin{itemize}
    \item A set $\mathcal{N}$ of $n$ indices of \textbf{non-basic variables} with value zero
    \item A set $\mathcal{B}$ of $n$ indices of \textbf{basic variables} whose values are then
        uniquely defined by the $m$ equations.
\end{itemize}

\subsection*{Why can't we just work with vertices?}

\begin{itemize}

    \item A vertex is also a point with $n$ zero components and $m$ non-negative components
        uniquely defined by $\overline{A}\textbf{x} = \textbf{b}$

    \item A \textbf{degenerate} vertex has more than $n$ zero components
        \begin{itemize}
            \item More than on partition of $\{1, 2, \ldots, n+m\}$ into sets $\mathcal{N}$ and
                $\mathcal{B}$ is possible
            \item There may be more than one basic solution at a degenerate vertex.
        \end{itemize}

\end{itemize}

\subsection*{Theorem 3: A sufficient optimality condition for LP problems}

A point $\textbf{x} \in K$ is an optimal solution of an LP problem if it is a basic feasible
solution with non-positive \textbf{reduced costs}
$\widehat{\textbf{c}}_N = \textbf{c}_N - N^{T}B^{-T}\textbf{c}_B \le 0$

\section*{The simplex algorithm}

\subsection*{Description of the simplex algorithm}

\begin{enumerate}

    \item If the reduced costs are non-positive then \textbf{stop} \\
        \textbf{The solution is optimal}

    \item Determine the non-basic variable $x_{q'}$ with the most positive reduced cost

    \item Determine the feasible direction $\textbf{d}$ when $x_{q'}$ is increased from zero

    \item If no basic variable is zeroed on $\textbf{x} + \alpha \textbf{d}$ then \textbf{stop} \\
        \textbf{The LP is unbounded}

    \item Determine the first basic variable $x_{p'}$ to be zeroed on
        $\textbf{x} + \alpha \textbf{d}$

    \item Make $x_{p'}$ non-basic and $x_{q'}$ basic

    \item Go to 1

\end{enumerate}

\subsection*{Definition of the simplex algorithm}
Given a basic feasible solution \textbf{x} with $\mathcal{B}$ and $\mathcal{N}$

\begin{enumerate}

    \item If $\widehat{\textbf{c}}_N \le \textbf{0}$ then \textbf{stop}
        (with $\widehat{\textbf{c}}_N = \textbf{c}_N - {N^T} B^{-T}\textbf{c}_B$)\\
        \textbf{The solution is optimal}

    \item Determine the index $q' \in \mathcal{N}$ of the variable $x_{q'}$ with the most
        positive reduced cost $\widehat{c}_q$ \\
        $q'$ is the $q\text{th}$ entry in $\mathcal{N}$.

    \item Let $\widehat{\textbf{a}}_q = B^{-1}\textbf{a}_q$,
        where $\textbf{a}_q$ is column $q$ of $N$

    \item If $\widehat{\textbf{a}}_q \le \textbf{0}$ then \textbf{stop} \\
        \textbf{The LP is unbounded}

    \item Determine the index $p' \in \mathcal{B}$ of the variable $x_{p'}$ corresponding to
        $p = \text{argmin}_{i=1}^m_{,\widehat{\textbf{a}}_{iq}>0}
        \frac{\widehat{b}_i}{\widehat{a}_{iq}}$ (with $\widehat{\textbf{b}} = B^{-1}\textbf{b}$)\\
        $p'$ is the $p\text{th}$ entry in $\mathcal{B}$

    \item Exchange indices $p'$ and $q'$ between $\mathcal{B}$ and $\mathcal{N}$
        to yield a new basic feasible solution

    \item Go to 1

\end{enumerate}

\subsection*{Obtaining the initial basic feasible solution}

As the initial basic feasible solution, try the ``all slack'' basis (i.e.\ starting at the origin)
\[
    \mathcal{B} = \{ n+1, \ldots, n+m \} \text{ and } \mathcal{N} = \{ 1, \ldots, n \}
\]
So we have:
\begin{itemize}
    \item $\widehat{\textbf{b}} = \textbf{b}$
    \item $\widehat{\textbf{c}}_N = \textbf{c}$
    \item Basis is feasible iff $\textbf{b} \ge \textbf{0}$
\end{itemize}

% \subsection*{Special cases and analysis}
\subsection*{How to start if $\textbf{b} \ngeq \textbf{0}$}\\
Can't use the ``all-slack'' basis (because the origin is not in the feasible region)

\begin{itemize}

    \item If $\textbf{b} \ngeq \textbf{0}$ then, for each constraint $i$, subtract an
        \textbf{artificial variable} $x_{n+m+i} \geq 0$

    \item Replace the objective $f = \overline{\textbf{c}}^T\textbf{x}$ with the \textbf{Phase I}
        objective $f = - \sum^m_{i=1} x_{n+m+i}$ (i.e.\ the negated sum of infeasibilities)

\end{itemize}

\subsection*{The Phase I problem}

Construct an initial basic feasible solution as follows: For $i = 1, \ldots, m$

If $b_i \geq 0$
\begin{itemize}
    \item Slack $x_{n+i} = b_i \geq 0$ is basic
    \item Artificial $x_{n+m+i} = 0$ is non-basic
    \item Column $i$ of $B$ is $\textbf{e}_i$
\end{itemize}

If $b_i < 0$
\begin{itemize}
    \item Slack $x_{n+i}$ is non-basic
    \item Artificial $x_{n+m+i} = -b_i > 0$ is basic
    \item Column $i$ of $B$ is $-\textbf{e}_i$
\end{itemize}

Basis matrix $B$ is non-singular and, by construction, $\widehat{\textbf{b}} \geq \textbf{0}$

\subsubsection*{At an optimal solution of the Phase I problem}

The simplex algorithm drives $f$ up towards zero.\\
At an optimal basic feasible solution $x$ of the Phase I problem:\\

If $f=0$
\begin{itemize}
    \item The values of the original and slack variables at $x$
        yield a basic feasible solution for the original LP
\end{itemize}

If $f<0$
\begin{itemize}
    \item The artificial variables cannot all be driven to zero
    \item The original LP is \textbf{infeasible}
\end{itemize}

If the Phase I problem is solved with $f=0$ (and all artificial variables being in $\mathcal{N}$)
\begin{enumerate}
    \item Remove the artificial variables from the problem (they are now zero)
    \item Revert to the original objective function
    \item Solve the original \textbf{Phase II} problem
\end{enumerate}

\subsubsection*{Does the algorithm terminate?}

\begin{itemize}
    \item If $\widehat{\textbf{b}}$ has any zero components then $\textbf{x}$ is a
        \textbf{degenerate} vertex
    \item There may be several basic feasible solutions at $\textbf{x}$
    \item If $\widehat{b}_p = 0$ then $\overline{\alpha} = 0$
        so the simplex algorithm does not move to a new vertex
    \item It may never leave!
\end{itemize}

\end{document}
